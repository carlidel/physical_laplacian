\documentclass[10pt,a4paper]{article}
\usepackage[utf8]{inputenc}
\usepackage[italian]{babel}

\usepackage{amsmath}
\usepackage{mathtools}
\usepackage{amsfonts}
\usepackage{amssymb}
\usepackage{amsmath}
\usepackage{siunitx}
\usepackage{physics}

\mathtoolsset{showonlyrefs,showmanualtags}

\usepackage{graphicx}
\usepackage{subfig}
\usepackage{wrapfig}
\usepackage{sidecap}
\usepackage{booktabs}
\usepackage{hyperref}

\newtheorem{theorem}{Theorem}[section]
\newtheorem{corollary}{Corollary}[theorem]
\newtheorem{lemma}[theorem]{Lemma}

%%% BackEnd Bibliografico
%\usepackage{textcomp}
%\usepackage[autostyle]{csquotes}
%\usepackage[
%        backend=biber,
%        %bibstyle=numeric,
%        %sorting=ynt
%    ]{biblatex}
%\addbibresource{bibliografia.bib}
%\nocite{*}
%%%

%%%%%% TESTO EFFETTIVO

\title{Relazione d'esame di Reti Complesse}
\author{Carlo Emilio Montanari}

\begin{document}

\maketitle

\tableofcontents

\begin{abstract}
    
\end{abstract}

\section{Nozioni di base}\label{sec:base}

Si esprime un grafo tramite la notazione \(G(V,E)\), dove \(V=\{1\ldots n\} \) è un insieme di \(n\) nodi ed \(E\) è un insieme di link.
Ad ogni link \(\langle i , j\rangle \) si associa un peso non negativo \(w_{ij}\), laddove due nodi non sono connessi con un link, si assume \(w_{ij}=0\).
Definiamo inoltre il vicinato di \(i\) come \(N(i)=\{j|\langle i,j \rangle \in E\} \) ed il \textit{degree} di un nodo \(i\) come \(\deg(i) = \sum_{j\in N(i)} w_{ij}\).
Nel presente lavoro, si assume \(G\) connesso (qualora si abbia a che fare con grafi non connessi, si hanno da trattare le componenti distinte separatamente).

La \textit{matrice delle adiacenze} di un grafo \(G\) è la matrice simmetrica \(n \times n\) definita come
\begin{equation}
    A_{ij}^G =
    \begin{cases}
        0, & i = j \\
        w_{ij} & i \ne j
    \end{cases}
    i,j = 1,\ldots,n
\end{equation}

Il \textit{Laplaciano} di un grafo è una matrice simmetrica \(n \times n\) definita come
\begin{equation}
    L_{ij}^G =
    \begin{cases}
        \deg(i), & i = j \\
        -w_{ij} & i \ne j
    \end{cases}
    i,j = 1,\ldots,n
\end{equation} 
Il Laplaciano è semi-definito positivo ed ha un solo autovalore nullo con associato autovettore \(1_n\).
Una caratteristica importante del Laplaciano è data dal seguente lemma:
\begin{lemma}\label{lem:laplaciano_quadratico}
    Sia \(L\) un Laplaciano \(n\times n\) e sia \(\vb{x}\in \mathcal{R}^n\). Allora si ha
    \begin{equation}
        \vb{x}^T L \vb{x} = \sum_{i<j} w_{ij}{(x_i - x_j)}^2
    \end{equation}
    La dimostrazione è immediata a partire dalla definizione di Laplaciano.
\end{lemma}
Da questo lemma consegue che la forma quadratica associata al Laplaciano è in pratica una somma pesata di ogni coppia di distanze.

Nel presente lavoro, si considerano gli autovalori di Laplaciano ordinati secondo la convenzione \(0 = \lambda_1 < \lambda_2 \leq \cdots \leq \lambda_n\) ed i corrispondenti autovettori reali ortonormali rappresentati tramite la notazione \(v_1 = (1/\sqrt{n})\cdot 1_n, v_2, \ldots, v_n\).

Definiamo la \textit{Degree Matrix} come la matrice \(n\times n\) diagonale \(D\) tale che \(D_{ii} = \deg(i)\).
Data una Degree Matrix \(D\) ed un Laplaciano \(L\), si dice che un vettore \(\vb{u}\) ed un scalare \(\mu \) sono una auto-coppia generalizzata di \(L,D\) se \(L\vb{u}=\mu D \vb{u}\).

\section{Spectral Drawing di un grafo}\label{sec:spectral_drawing}

\section{Perturbazioni e modifiche del Laplaciano}\label{sec:perturbazioni}

\section{Applicazione a Toy Models}\label{sec:applicazione}



\end{document}